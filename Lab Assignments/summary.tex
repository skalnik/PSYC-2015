\documentclass[12pt]{article}
\title{Summary}
\author{Mike Skalnik}

\renewcommand{\baselinestretch}{2}
\usepackage[left=.5in,top=.5in,right=.5in,bottom=.5in,nohead]{geometry}

\begin{document}
\begin{flushright}{\large Summary\\ Mike Skalnik}\end{flushright}

Darwinian evolution has been witnessed as a process that will accomplish an end goal, even if the outcome is not what could be expected. Due to this, it has become more and more common to simulate evolution to study various topics. I would like to investigate the evolution of a functioning timepiece out of parts. Wirt Atmar (1994) outlines the general idea of how such a program would be written. An initial population is generated, then replicated with random mutations introduced, the new population is judged by a set of criteria, and any unacceptable candidates are removed with acceptable candidates kept, and then the cycle starts again. This continues until a candidate reaches a set of predetermined rules that are used to define an acceptable solution.

While there have been rather complex simulations, such as the evolution of highly divergent DNA sequences by Strope, Abel, Scott, and Moriyama (2009), studying the evolution of timepieces would give more insight into moderately complex sequences. I would create such a simulation, tweak the rules that define the relationship between the various watch parts, run the simulation, and see how many generations it takes before an acceptable watch is created.

I predict that a basic affinity between parts and fitness based upon accuracy is all that is required to create a working timepiece. Exactly what the affinity rules would be are hard to determine at this point. However, the work of Hase, Khang, and Eom (2004), who simulated evolution to model hopping motions, can be used to determine good initial values. Once optimal rules have been discovered, the initial population can be changed as well to see how the better watches turn out. This would also be interesting since one could examine the difference watches created with this process and watches in real life. I predict that the watches created by such a process would be radically different than the ones we have made due to the restrictions that have been worked around. However, if these rules that model these restrictions are implemented, then I feel like the outcome would be relatively close to watches created by humans.

\end{document}
